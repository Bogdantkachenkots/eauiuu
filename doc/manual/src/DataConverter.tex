% !TEX root = EUDAQUserManual.tex
\section{Writing a Data Converter}
As a framework, EUDAQ itself is develped with no knowledge of how hardware data can be decoded. Only the user can know the specific detail of the readout data from hardware, so users are required to write a data converter derived from eudaq::DataConverter to convert to other format. The converted data then can be stored on disk or used as input data of any non-EUDAQ software.\\

Currently, as historical legacy, there are two different formats are provided along with the native raw eudaq::Event. They are eudaq::StandardEvent for pixel detector, and eudaq::LCEvent as an EUDAQ wrapper for LCIO conntianter.

\subsection{Event Structure}
eudaq::Event is the most important data containter in EUDAQ system. eudaq::Event should be fillied by detector data porperly in order to transmit among eudaq clients, eg. Producer and DataCollector. \autoref{tab:eventvarialbe} lists all the member variables inside of eudaq::event. Not all of these variable are mandated to be filled. For example, in case there is tirgger number but no timestamp information, the m\_ts\_begin and m\_ts\_end can be left untouched.

\begin{table}[!h]
{\footnotesize
\begin{tabular}{l|l|p{2cm}|p{5.5cm}}
Variable & Type & Default Value & Comment \\
\hline
m\_type & uint32\_t & - & Event type\\
m\_version & uint32\_t & - & Version of EUDAQ\\
m\_flag & uint32\_t & - & Event flag\\
m\_stm\_n & uint32\_t & - & Stream/device number\\
m\_run\_n & uint32\_t & - & Run number\\
m\_ev\_n & uint32\_t & - & Event number\\
m\_tg\_n & uint32\_t & - & Trigger number\\
m\_extend & uint32\_t & - & Reserved word\\
m\_ts\_begin & uint64\_t & - & Timestamp at the begin of trigger\\
m\_ts\_end & uint64\_t & - & Timestamp at the end of trigger\\
m\_dspt & std::string & - & Description String\\
m\_tags & std::map$<$std::string, std::string$>$& - & Tag pairs\\
m\_blocks & std::map$<$uint32\_t, std::vector$<$uint8\_t$>>$ & - & Raw data blocks\\
m\_sub\_events & std::vector$<$EventSPC$>$& - & Sub events\\
\end{tabular}
\caption{all member variables in eudaq::Event}
\label{tab:eventvarialbe}
}
\end{table}

The eudaq::Event is also capable to contain sub event inside of itself. 

\subsubsection{RawDataEvent}\label{sec:convrawdata}
eudaq::RawDataEvent is derived from eudaq::Event with the m\_type always set to eudaq::cstr2hash("RawDataEvent"). There is no more member variables or methods than the base eudaq::Event. However the memver variable m\_extend is used as a identification number of sub type of event.

\subsubsection{StandardEvent}\label{sec:convstddata}
eudaq::StandardEvent is derived from eudaq::Event with the m\_type always set to eudaq::cstr2hash("StandardEvent"). Historily, it presents a beam telescope tracking-hit event with all hit information of sensor planes. The hit information is contained by eudaq::StandardPlane. The beam telescope planes are pixel sensors. The spatial position and signal amplitude of the fired pixels are Zero-Compressed inside eudaq::StandaredPlane.

\subsection{Example Code: RawEvent2StdEvent}\label{sec:Ex0RawEvent2StdEventConverter_cc}
This example DataConverter is named Ex0RawEvent2StdEventConverter. As the hint by the name, it takes responsible to convert the eudaq::RawDataEvent to eudaq::StandardEvent. The sub type of eudaq::RawDataEvent is "my\_ex0" which also used to caculate the hash and retester to EUDAQ factory. If an eudaq::RawDataEvent object announcing its sub-type by "my\_ex0" exists when doing the data converting, this object will be forwarded to that Ex0RawEvent2StdEventConverter.
\lstinputlisting[label=ls:ex0raw2std, style=cpp]{../../user/example/module/src/Ex0RawEvent2StdEventConverter.cc}

