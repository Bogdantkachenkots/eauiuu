% !TEX root = EUDAQUserManual.tex
\section{Writing a Producer}\label{sec:ProducerWriting}
Producers are the binding part between a user DAQ and the central EUDAQ Run Control. The base class of Producer is eudaq::Producer. The eudaq::Producer is an inherited class of eudaq::CommandReceiver which receives commands from the eudaq::RunControl. It also maintaines a set of eudaq::DataSender which allows to send binary data (eudaq::Event) to each destinations.\\

\subsection{Declaration of base Producer}\label{sec:Producer_hh}

\autoref{producerdef} is part of the header file which declares the eudaq::Producer. You are aquired to write the user Producer derived from eudaq::Producer. According the the set of commands (\autoref{sec:command}) in EUDAQ framwork, there are 5 virtual functions should be implemented in user producer. They are DoInitialise(), DoConfigure(), DoStopRun(), DoStartRun(), DoReset() and DoTerminate(). These command related function should return as soon as possiable since no further command can be execuseted untill the finish the recent command. \\

\lstinputlisting[label=producerdef, style=cpp, linerange=BEGINDECLEAR-ENDDECLEAR]{../../main/lib/core/include/eudaq/Producer.hh}

The virtual function Exec() is optional to be implimented in user producer. Internally, the base Exec() to create a thread for command executation and itself goes to a unlimited loop and can never return untill the Terminate command. In case you are going to impliment it yourselves, please read the souce code to find detial information.

\subsection{Example Source Code}\label{sec:Ex0Producer_cc}
\subsubsection{Declaration}
Let's go to a example implemention of user Producer. The full source code is avaliable at file path user/example/module/src/Ex0Producer.cc.  \autoref{ls:ex0producerdef} is the declaration part. There are 6 commmand related functions, a constructor function and a \texttt{Mainloop} function. \\

\lstinputlisting[label=ls:ex0producerdef, style=cpp, linerange=BEG*DEC-END*DEC]{../../user/example/module/src/Ex0Producer.cc}

In the line \lstinline[style=cpp]{static const uint32_t m_id_factory = eudaq::cstr2hash("Ex0Producer")}, it defines a static number wich is a hash from name string. This number will be used later as a ID to register this \lstinline[style=cpp]{Ex0Producer} to EUDAQ runtime env.

\subsubsection{Register to Factory}
To make EUDAQ framework know the eiisting of \lstinline[style=cpp]{Ex0Producer}, this Producer should regitested to \lstinline[style=cpp]{eudaq::Factory<eudaq::Producer>} with it static ID number. The input paramter types of construtor function is also provided to Register function. See the example code below.
\lstinputlisting[label=ls:ex0producerdef, style=cpp, linerange=BEG*REG-END*REG]{../../user/example/module/src/Ex0Producer.cc}

\subsubsection{Constructor}
The Constructor function takes in 2 parameters, the runtime name of the instance of Producer and the address of RunControl. 
\lstinputlisting[style=cpp, linerange=BEG*CON-BEG*INI]{../../user/example/module/src/Ex0Producer.cc}
In this example of Ex0Producer, constructor function does nothing beside passing pamamters to base constructor function of Producer and inilialize the \lstinline[style=cpp]{m_exit_of_run} variable.

\subsubsection{DoInitialise}
This method is called whenever an initialise command is received from the Run Control. When This function is called, the correlated section of initializition file have arrived to the Producer from RunControl. This initializition section can be obtained by \\
\lstinline[style=cpp]{eudaq::ConfigurationSPC eudaq::CommandReceiver::GetInitConfiguration()} \\

Here is example \lstinline[style=cpp]{DoInitialise} of Ex0Producer.
\lstinputlisting[style=cpp, linerange=BEG*INI-BEG*CONF]{../../user/example/module/src/Ex0Producer.cc}
The Configuration object \lstinline[style=cpp]{ini} is obtained. According to the value DMMMY\_STRING and UMMY\_FILE\_PATH in \lstinline[style=cpp]{ini} object, a new file is openned and filled by dummy data. The path of file is saved as variable member for later access of this dummy data file.

\subsubsection{DoConfigure}
This method is called whenever an configure command is received from the Run Control. When This function is called, the correlated section of configuration file have arrived to the Producer from RunControl. The configuration section named by Procuder runtime name can be obtained by \\
\lstinline[style=cpp]{eudaq::ConfigurationSPC eudaq::CommandReceiver::GetConfiguration()} \\

Here is example \lstinline[style=cpp]{DoConfigure} of Ex0Producer.
\lstinputlisting[style=cpp, linerange=BEG*CONF-BEG*RUN]{../../user/example/module/src/Ex0Producer.cc}
The variable \lstinline[style=cpp]{m_ms_busy}, \lstinline[style=cpp]{m_flag_ts} and \lstinline[style=cpp]{m_flag_tg} are set according the Configuration object.

\subsubsection{DoStartRun}\label{sec:ex0prdstart}
This method is called whenever an StartRun command is received from the Run Control. When This function is called, the runnumber have already been increased by 1. For real hardware specific Producer, the hardwre is told to starup.
Here is example \lstinline[style=cpp]{DoStartRun} of Ex0Producer.
\lstinputlisting[style=cpp, linerange=BEG*RUN-BEG*STOP]{../../user/example/module/src/Ex0Producer.cc}
Instead of talking to real hardware, a file is opened. Then, a new thread is started using the function \lstinline[style=cpp]{Mainloop}.

\subsubsection{DoStopRun}
This method is called whenever an StopRun command is received from the Run Control. Here is example \lstinline[style=cpp]{DoStopRun} of Ex0Producer.
\lstinputlisting[style=cpp, linerange=BEG*STOP-BEG*RST]{../../user/example/module/src/Ex0Producer.cc}

\subsubsection{DoReset}
When the Producer goes into the error state. Only Reset command is acceptable. It is recommand to reset all member varialbe to original value when the proucer object is instanced.\\
Here is example \lstinline[style=cpp]{DoReset} of Ex0Producer. 
\lstinputlisting[style=cpp, linerange=BEG*RST-BEG*TER]{../../user/example/module/src/Ex0Producer.cc}
For any case the data thread is running, it should be stopped.

\subsubsection{DoTerminate}
This method is called whenever an StopRun command is received from the Run Control. After the return of \lstinline[style=cpp]{DoTerminate}, the application will exit.\\
Here is example \lstinline[style=cpp]{DoStopRun} of Ex0Producer.
\lstinputlisting[style=cpp, linerange=BEG*TER-BEG*LOOP]{../../user/example/module/src/Ex0Producer.cc}
The data thread is closed. 

\subsubsection{Send Event}
In \lstinline[style=cpp]{DoStartRun}, a data thread is created with the function \lstinline[style=cpp]{Mainloop}. Here is the implemention of this function. The generated data Event is depending the value set by \lstinline[style=cpp]{DoInitialise} and \lstinline[style=cpp]{DoConfigure}.
\lstinputlisting[style=cpp, linerange=BEG*LOOP-END*IMP]{../../user/example/module/src/Ex0Producer.cc}
In each loop, a new object of Event named Ex0Event is created. Trigger number and Timestamp are optional to be set depending the flags. A data block with 100 zeros is filled to the Event object by id 0. Another data block read from file is filled to some Event object by id 2.Then, this Event object is sent out by \lstinline[style=cpp]{SendEvent(std::move(ev))}. The first Event object has a flag BORE by method \lstinline[style=cpp]{eudaq::Event::SetBORE()}

\paragraph{Tags}\label{sec:Tags}
The \texttt{Event} class also provide the option to store tags.
Tags are name-value pairs containing additional information which does not qualify as regular DAQ data which is written in the binary blocks.
Particularily in the \gls{BORE} this is very useful to store information about the exact sensor configuration which micht be required in order to be able to decode the raw data stored.
A tag is stored as follows:
\begin{listing}
event.SetTag("Temperature", 42);
\end{listing}

The value corresponding to the tag can be set as an arbitrary type (in this case an integer),
it will be converted to a STL string internally.

\subsubsection{Error}\label{sec:Tags}
The the case Producer fail to run the command function. Throw an execption like this\\
\lstinline[style=cpp]{EUDAQ_THROW("dummy data file (" + m_dummy_data_path +") can not open for writing")}\\
Since Mainloop function is running as an independent thread, the execption throw from it can not catched from outside.

