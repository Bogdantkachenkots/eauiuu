% !TEX root = EUDAQUserManual.tex
\section{Handling New Data Type}\label{sec:NewDataHandling}
The EUDAQ framework contains a number of other parts that may be useful.
Those that have not already been described in previous sections will be outlined below.

\subsection{Derive from Event}


\subsection{Writing a FileWriter}\label{sec:FileWriterWriting}
The FileWriter part of the framework allows different file formats to be written,
using a common interface, using a plugin-like system to define new file types.
The \texttt{FileWriter} class defines the interface that each type must implement,
and the \texttt{FileWriterFactory} allows code that writes data files to select any
available file type, and write it in a generic way,
without needing to know details about the particular file format.
A number of different file types are already implemented,
for a list with descriptions, see page \pageref{lst:FileTypes}

The easiest way to make use of the different \texttt{FileWriter}s,
is to use the \texttt{Converter.exe} program (see \autoref{sec:Converter}).

The \texttt{FileWriter} base class is defined in the following header:
\begin{listing}
#include "eudaq/FileWriter.hh"
\end{listing}

In order to implement a new \texttt{FileWriter}, a new class must be written,
inheriting from the \texttt{FileWriter} base class, and implementing the following methods:
\begin{listing}
virtual void StartRun(unsigned);
virtual void WriteEvent(const DetectorEvent &);
virtual uint64_t FileBytes() const;
\end{listing}

The \texttt{StartRun} method is called at the start of each new run
with the run number as a parameter.
This allows a new file to be opened, and any header information to be written if necessary.
Then the \texttt{WriteEvent} method is called for each event to be written.
Here the \texttt{DetectorEvent} can be decoded and processed
and the necessary data written to file.
The \texttt{FileBytes} method should return the number of bytes written to the file.
However, it is optional, and may simply return zero if the actual size is not easily known.

\subsection{Writing a FileReader}\label{sec:FileReaderWriting}
Although tools are provided to access the information in the native data files,
and to convert them to other formats (such as LCIO, for analysis with the EUTelescope package),
in some cases it may be preferable to access the native data directly.
For this, the \texttt{FileReader} class is provided,
allowing a custom program to be written to access a native file and process it as desired.

The constructor takes as an argument the name of the file to be opened,
and will read the first event from the file (which should be the \gls{BORE}).
The \texttt{NextEvent()} method can then be called to advance through the file.
It can optionally take as a parameter the number of events to skip,
and will return \texttt{true} as long as a new event was read.
The currently loaded event can be accessed with the \texttt{GetDetectorEvent()} method.

The basic usage is shown below, while a more complete example is available in \autoref{sec:ExampleReader}:
\begin{listing}
#include "eudaq/FileReader.hh"
#include <iostream>

int main(int argc, char ** argv) {
  if (argc != 2) {
    std::cerr << "usage: " << argv[0] << " file" << std::endl;
    return 1;
  }
  eudaq::FileReader reader(argv[1]);
  std::cout << "Opened file: " << reader.Filename() << std::endl;
  std::cout << "BORE:\n" << reader.GetDetectorEvent() << std::endl;
  while (reader.NextEvent()) {
    std::cout << reader.GetDetectorEvent() << std::endl;
  }
  return 0;
}
\end{listing}

This will open the file specified on the command line,
and print out a summary of all the events in there.
Be aware that running it as it is may generate a large amount of output,
especially with large data files.

\subsection{Writing a Data Converter (base class)}\label{sec:BaseConverterWriting}
The \texttt{PluginManager} handles the different \texttt{DataConverterPlugin}s,
allowing raw data stored in a \texttt{RawDataEvent} to be easily converted
to a \texttt{StandardEvent} or \texttt{LCEvent} without having to know the details of all the detector types in there.
It is defined in the following header:
\begin{listing}
#include "eudaq/PluginManager.hh"
\end{listing}

In order to convert the events correctly,
the plugins must have access to the information in the BORE.
Therefore, before any events may be converted, and for each data file,
the \texttt{PluginManager} must be initialized as follows:
\begin{listing}
eudaq::PluginManager::Initialize(bore);
\end{listing}

The \texttt{PluginManager} will take care of passing the relevant parts of the \gls{BORE}
to the appropriate \texttt{DataConverterPlugin}s.
The \texttt{DetectorEvent}s can then be converted as follows:
\begin{listing}
eudaq::StandardEvent sev = eudaq::PluginManager::ConvertToStandard(dev);
\end{listing}

The \texttt{PluginManager} will take care of splitting the \texttt{DetectorEvent}
into its constituent subevents, and passing them all to the appropriate
\texttt{DataConverterPlugin}s to be inserted into the returned \texttt{StandardEvent}.
For a slightly more complete example of the \texttt{PluginManager},
see the \texttt{ExampleReader} in \autoref{sec:ExampleReader}.

