% !TEX root = EUDAQUserManual.tex
\section{Support User Defined Event Type}\label{sec:NewDataHandling}
In EUDAQ framework, the definition \texttt{data} usually means a object of eudaq::Event or its derivations. 

\subsection{Event}
eudaq::Event is serializable as it is derived from eudaq::Serializable and implements
\begin{listing}
  Event(Deserializer &);
  virtual void Serialize(Serializer &) const;    
\end{listing}

There are 3 eudaq::Event derivations existing by default installation of EUDAQ. They are \lstinline[style=cpp]{RawDataEvent} (\autoref{sec:convrawdata}), \lstinline[style=cpp]{StandardEvent} (\autoref{sec:convstddata}) and \lstinline[style=cpp]{LCEvent}. Certainly, the eudaq::Event derivations are serializable. But if there are additional member variable intoruded to derived Event, the new variable should be serialized along with the other variables defined in eudaq::Event. Otherwise, part data of user defined Event object are losted when doing the Event writing to disk by EUDAQ native format (\autoref{sec:FileWriterWriting}).\\

It is recommend to reuse the raw data blocks \lstinline[style=cpp]{m_blocks} of eudaq::Event. Then the base eudaq::Event will take care of serialize and deserialize the \lstinline[style=cpp]{m_blocks}.\\

\subsection{FileWriter}\label{sec:FileWriterWriting}
Usually, the objects of eudaq::Event and its derivations are the only data going to be stored in disk for later offline analysis. User may prefer to write Event data in other format which is human readable or widely used by other software. User can defined a class derived from eudaq::FileWriter to take response the writing of specific Event type.

\subsubsection{FileWriter Prototype}
\lstinputlisting[label=ls:filewritedec, style=cpp, linerange=BEG*DEC-END*DEC]{../../main/lib/core/include/eudaq/FileWriter.hh}

\subsubsection{Example Code: NativeFileWriter}
No matter if the user defined Event reuses the \lstinline[style=cpp]{m_blocks} raw data block or does serialization of new variables itself, it can be write to disk in binary format (aka native) which can only be read by EUDAQ.

\lstinputlisting[label=ls:rawfilewritecc, style=cpp]{../../main/module/std/src/RawFileWriter.cc}

\subsection{FileReader}\label{sec:FileReaderWriting}
To reconstruct the Event from disk file in native format or user-defined format, class eudaq::FileReader is provided. Each writing data format should have its correlated implementation of FileReader to access the disk file.

\subsubsection{FileReader Prototype}
\lstinputlisting[label=ls:filereaderdec, style=cpp, linerange=BEG*DEC-END*DEC]{../../main/lib/core/include/eudaq/FileReader.hh}

\subsubsection{Example Code: NativeFileReader}
\lstinputlisting[label=ls:rawfilereadercc, style=cpp]{../../main/module/std/src/RawFileReader.cc}
