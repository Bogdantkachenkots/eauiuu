% !TEX root = EUDAQUserManual.tex
\section{Contributing to EUDAQ}
\label{sec:contributing}

\subsection{Reporting Issues}
\label{sec:reporting}
The GitHub server, on which EUDAQ is hosted, provides a system for reporting bugs and for requesting new features.
It is accessible at the following address: 

\url{https://github.com/eudaq/eudaq/issues}.

Here you may submit new reports (you are required to register first to do this),
or follow the status of existing bugs and feature requests.
This is recommended over (or at least, as well as) sending an email to the developers,
as it ensures a record of the issue is available, and others may follow the progress.

\subsection{Regression Testing}
\label{sec:ctest}

If a CMake version later than 3.1 is found and Python is installed
together with the \texttt{numpy} package, several regression tests are made
available that can be executed through CTest. The tests are based on
the Python wrapper around EUDAQ components as described in
section~\ref{sssec:pywrapper}. Run the tests by typing

\begin{listing}[mybash]
  cd build
  cmake ..
  ctest
\end{listing}

This starts the script \texttt{etc/tests/run\_dummydataproduction.py}
which runs a short DAQ session with instances of RunControl,
DataCollector and a (dummy) Producer and compares the output to a
reference file stored on AFS at DESY. If your system is set up
correctly, you have access to the reference file, and the basic
components of the EUDAQ library work, all tests should pass.
To see the output of failing tests, you can add the
\texttt{--output-on-failure} parameter to the CTest command.

These basic tests can easily be extended to test other parts of the
core framework or of your own producer. Take a look at the
\texttt{etc/tests/testing.cmake} CMake script and the central
\texttt{CMakeLists.txt} file where it is included for an example of
how to set up tests with CTest.

The automated nightly tests are defined in CMake scripts located in
\texttt{etc/tests/nightly} and are executed by the scripts
\texttt{run\_nightly.sh} and \texttt{run\_nightly.bat} for Unix and
Windows platforms, respectively. In addition to the dummy run
described above, the nightly tests check out all changes from the
central repository, build the full code base, and submit all results
to the CDash webserver hosted at DESY: \url{http://aidasoft.desy.de/CDash/index.php?project=EUDAQ}

\subsection{Commiting Code to the Main Repository}
\label{sec:commiting}



If you would like to contribute your code back into the main repository, please follow the ``fork \& pull request'' strategy:

\begin{itemize}
\item Create a user account on github, log in
\item ``Fork'' the (main) project on github (using the button on the page of the main repo)
\item \emph{Either}: clone from the newly forked project and add
  'upstream' repository to local clone (change user names in URLs
  accordingly):
  \begin{listing}[mybash]
git clone https://github.com/your_account_name/eudaq eudaq
cd eudaq
git remote add upstream https://github.com/eudaq/eudaq.git
\end{listing}
\item \emph{or} if edits were made to a previous checkout of upstream: rename origin to upstream, add fork as new origin:

  \begin{listing}[mybash]
cd eudaq
git remote rename origin upstream
git remote add origin https://github.com/your_account_name/eudaq
git remote -v show
\end{listing}
\item Optional: edit away on your local clone! But keep in sync with
  the development in the upstream repository by running
  \begin{listing}
git fetch upstream        # download named heads or tags
git pull upstream master  # merge changes into your branch
\end{listing}
on a regular basis. Replace \texttt{master} by the appropriate branch if you work on a separate one.
Don't forget that you can refer to issues in the main repository anytime by using the string \texttt{eudaq/eudaq\#XX} in your commit messages, where \texttt{XX} stands for the issue number, e.g.
  \begin{listing}[mybash]
[...]. this addresses issue eudaq/eudaq#1
\end{listing}
\item Push the edits to origin (our fork)
  \begin{listing}[mybash]
git push origin
\end{listing}
(defaults to \texttt{git push origin master} where origin is the repo and master the branch)
\item Verify that your changes made it to your github fork and then click there on the ``compare \& pull request'' button
\item Summarize your changes and click on ``send''
\item Thank you!
\end{itemize}


Working together on a branch:
If you have a copy installed, and want to update it to the
latest version, you do not need to clone the repository again, just change to the \texttt{eudaq} directory use the command:
\begin{listing}[mybash]
git pull
\end{listing}
to update your local copy with all changes commited to the central repository.%

