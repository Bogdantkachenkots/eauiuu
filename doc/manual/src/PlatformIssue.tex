% !TEX root = EUDAQUserManual.tex
\section{Platform Dependent Issues/ Solutions}
\label{app:platform}
\subsection{Linux}
TODO
\subsection{MacOS}
TODO
\subsection{Windows}
\subsubsection{MSBUILD}

This is the program that processes the project (solution) files and feeds it to the compiler and linker. If you have a working project file it is more or less straight forward. It has a very simple syntax:

      \begin{listing}[mybash]
$[MSBUILD.exe]$ MyApp.sln /t:Rebuild /p:Configuration=Release
\end{listing}

myApp.sln is the file you want to process. The parameter
\texttt{/target} (short \texttt{/t}) tells msbuild what to do, in this
case rebuild. You have all the options you need like: clean, build and
rebuild. You can also specify your own targets. With the ``parameter
property'' switch you can change the properties of your Project. If you want to compile EUDAQ, you go in the build folder where the solution (sln) file is and type:

      \begin{listing}[mybash]
$[MSBUILD.exe]$ EUDAQ.sln /p:Configuration=Release 
\end{listing}

One thing one has to keep in mind is that there are some default
configurations. The default is a debug build for x86. If you want a
different build then you need to specify it in the command line. And
one thing you want to have is a release build! With the /p switch you
can overwrite properties like in this case the configuration. But you
could also overwrite the compiler version it should use. For example, if you
want to use VS 2013 then you have to specify it by writing:

      \begin{listing}[mybash]
$[MSBUILD.exe]$ EUDAQ.sln /p:PlatformToolset=v120 /p:Configuration=Release
\end{listing}

But be careful when changing the compiler settings. It is possible that some then link against an incompatible version of your external libraries. 

\subsubsection{Known Problems}

\begin{itemize}
\item The environment variables are pulled in as properties therefore they can be overwritten in the project file or in the ``vcxproj.user'' file. So if for example your QT Project does not compile and keeps complaining about not finding the correct directory make sure you are not overwriting the QTDIR environment Variable with a Property. 
\end{itemize}
